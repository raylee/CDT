\documentclass{article}

\title{Sphere Generator: Techniques}
\author{Jonah Miller\\\textit{jonah.maxwell.miller@gmail.com}}

\begin{document}

\maketitle

This is documentation for sphere\_generator.py, a program to generate
spheres of arbitrary surface area triangulated by equilateral
triangles. The sphere is represented as a list of triples, each triple
containing 3 vertex id numbers. Matching up the IDs yields an object
homeomorphic to a sphere, and with an (ideally) uniform curvature.

This is a description of the ideas used to make the sphere generator
program. The README and the programmer's guide offer more details. In
this document I will describe how the program selects for standard
deviation surface area of a given sphere.

Associated with each vertex is a point curvature:
$$K_p(v) = 2(2 - \theta_t N_T(v)),$$
where $\theta_t = \pi/3$ is the interior angle of an equilateral
triangle and $N_t(v)$ is the total number of triangles connected to a
given vertex. This description of curvature is given by Regge
calculus.

The mean curvature is then 
$$K = \frac{1}{N_v}\sum_{vertices, \ v} K_p(v),$$
where $N_v$ is the total number of vertices in the sphere.

The standard deviation of curvature, $\sigma$ is then given by
$$\sigma^2 = \frac{1}{N_v} \sum_{vertices, \ v}\left(K - K_p(v)\right)^2.$$

Let $\epsilon_a$ and $\epsilon_\sigma$ be numbers between 0 and 1. Let
the target standard deviation---i.e., the standard deviation we want
the sphere to have---be $\sigma_t$ and the target surface area of the
sphere---i.e., the surface area we want the sphere to have---be
$A_t$. The \textit{fitness function} for a given sphere $S$ is
$$f(S) = e^{-\epsilon_a (A - A_t)}e^{-\epsilon_\sigma (\sigma - \sigma_t)},$$
where $A$ is the surface area of the sphere at a given time. 

The metropolis algorithm does the following:
\begin{itemize}
\item Make a small change to the sphere using one of the ergodic
  moves. Ergodic here means that a composition of changes can bring
  any sphere to any other sphere.
\item Test to see if the fitness function becomes larger or
  smaller. If the fitness function becomes larger, accept the
  change. Otherwise, reject it.
\item Repeat.
\end{itemize}

That's it! That's how spheres are generated! For implementation
details, see the programmer's guide. If you have any questions, feel
free to email me.

\end{document}