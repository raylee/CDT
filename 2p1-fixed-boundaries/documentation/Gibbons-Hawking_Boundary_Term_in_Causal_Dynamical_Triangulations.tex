\documentclass{article}

\title{On The Einstein-Hilber Action with the Gibbons-Hawking Boundary Term in Causal Dynamical Triangulations}
\author{Jonah M. Miller
 \and 
Joshua H. Cooperman}


%packages
\usepackage{fullpage}
\usepackage{amsmath}
\usepackage{amssymb}
\usepackage{mathptmx}
\usepackage{multicol}
\usepackage{subfigure}
\usepackage{tabularx}
\usepackage{booktabs}
\usepackage{graphicx}
\usepackage{latexsym}
\usepackage{mathrsfs}

\begin{document}

\maketitle

\section{Introduction}
For a $(2+1)$-dimensional spacetime manifold $\mathcal{M}$ with
boundary $\partial\mathcal{M}$, we must add to the Einstein-Hilbert
action,
\begin{equation}\label{EHaction}
S_{EH}[\mathbf{g}]=\frac{1}{16\pi G}\int_{\mathcal{M}}\mathrm{d}^{3}x\,\sqrt{-g}\left(R-2\Lambda\right),
\end{equation}
the Gibbons-Hawking boundary term,
\begin{equation}
S_{GH}[\mathbf{\gamma}]=\frac{1}{8\pi G}\int_{\partial\mathcal{M}}\mathrm{d}^{2}y\sqrt{|\gamma|}K.
\end{equation}
Here, $\mathbf{\gamma}$ is the induced metric on the boundary
$\partial\mathcal{M}$, and $K$ is the trace of the extrinsic curvature
of the boundary $\partial\mathcal{M}$. Regge demonstrated that, for a
triangulated spacetime manifold $\mathcal{T}$, the Einstein-Hilbert
action assumes the form
\begin{equation}\label{Reggeaction}
S_{EH}^{(R)}[\mathcal{T}]=\frac{1}{8\pi G}\sum_{h\in\mathcal{T}}A_{h}\delta_{h}-\frac{\Lambda}{8\pi G}\sum_{s\in\mathcal{T}}V_{s}.
\end{equation}
Here, $h$ is a $1$-dimensional hinge having area $A_{h}$ and deficit
angle $\delta_{h}$, and $V_{s}$ is the spacetime volume of a
$3$-simplex $s$. Hartle and Sorkin demonstrated that, for a
triangulated spacetime manifold $\mathcal{T}$ with boundary
$\partial\mathcal{T}$, the Gibbons-Hawking boundary term assumes the
form
\begin{equation}
S_{GH}^{(R)}[\partial\mathcal{T}]=\frac{1}{8\pi G}\sum_{h\in\partial\mathcal{T}}A_{h}\psi_{h}.
\end{equation}
Here, $h$ is a $1$-dimensional hinge on the boundary
$\partial\mathcal{T}$ having area $A_{h}$, and $\psi_{h}$ is the angle
between the two vectors normal to the two spacelike $2$-simplices
intersecting at the hinge $h$.

We wish to determine the form of the Regge-Einstein-Hilbert action
supplemented by the Regge-Gibbons-Hawking boundary term in
$(2+1)$-dimensional causal dynamical triangulations for two-sphere
spatial topology and line interval temporal topology. Ambj\o rn
\emph{et al} \cite{DynamicallyTriangulating} give the Regge-Einstein-Hilbert action in this case,
finding that
\begin{eqnarray}
S_{EH}^{(R)}[\mathcal{T}_{c}]&=& k\left[\frac{2\pi}{i}N_1^{SL} - \frac{3}{i}k\theta_{SL}^{(3,1)}\left(N_3^{(3,1)} + N_3^{(1,3)}\right) - \frac{2}{i}\theta_{SL}^{(2,2)}N_3^{(2,2)}\right]\\
&+& k\sqrt{\alpha}\left[2\pi N_1^{TL} - 3\theta_{TL}^{(3,1)}(N_3^{(3,1)}+N_3^{(1,3)})-4\theta_{TL}^{(2,2)}N_3^{(2,2)}\right]\nonumber\\
&-& \frac{\lambda}{12}\left[(N_3^{(3,1)}+N_3^{(1,3)})\sqrt{3\alpha+1} + N_3^{(2,2)}\sqrt{4\alpha+2}\right]\nonumber
\end{eqnarray}
Here, $k = \frac{1}{8\pi G}$ and $\lambda = \Lambda k$ are coupling
constants. $N_{1}^{SL}$ is the number of spacelike $1$-simplices,
$N_{1}^{TL}$ is the number of timelike $1$-simplices, $N_{3}^{(2,2)}$
is the number of $(2,2)$ $3$-simplices, $N_{3}^{(1,3)}$ is the number
of $(1,3)$ $3$-simplices, $N_{3}^{(3,1)}$ is the number of $(3,1)$
$3$-simplices, $\theta_{SL}^{(2,2)}$ is the Lorentzian dihedral angle
about a spacelike edge of a $(2,2)$ $3$-simplex, $\theta_{TL}^{(2,2)}$
is the Lorentzian dihedral angle about a timelike edge of a $(2,2)$
$3$-simplex, $\theta_{SL}^{(1,3)}$ is the Lorentzian dihedral angle
about a spacelike edge of a $(1,3)$ $3$-simplex, $\theta_{TL}^{(1,3)}$
is the Lorentzian dihedral angle about a timelike edge of a $(1,3)$
$3$-simplex, $\theta_{SL}^{(3,1)}$ is the Lorentzian dihedral angle
about a spacelike edge of a $(3,1)$ $3$-simplex, $\theta_{TL}^{(3,1)}$
is the Lorentzian dihedral angle about a timelike edge of a $(3,1)$
$3$-simplex, $V_{3}^{(2,2)}$ is the spacetime volume of a $(2,2)$
$3$-simplex, $V_{3}^{(1,3)}$ is the spacetime volume of a $(1,3)$
$3$-simplex, $V_{3}^{(3,1)}$ is the spacetime volume of a $(3,1)$
$3$-simplex, and $\alpha$ is the ratio of the timelike to the
spacelike squared edge length of a $3$-simplex.The first line comes
from the summation over spacelike hinges, the second line comes from
the summation over timelike hinges, and the third line comes from the
summation over $3$-simplices. Note that we could rewrite the second
term of the first line using the relation
$4N_{1}^{SL}=3\left(N_{3}^{(1,3)}+N_{3}^{(3,1)}\right)$.

\section{The Bulk Action}

If the Regge-Einstein-Hilbert action for $(2+1)$-dimensional causal
dynamical triangulations (CDT) is to include a boundary term, it must
be split into a bulk term and a boundary term. Hartle and Sorkin
provide the framework for the Gibbons-Hawking-York boundary term
\cite{BoundaryTerm}. However, first, we must ensure that the
Regge-Einstein-Hilbert action does not take any boundary terms as
input. We will rederive the Regge-Einstein-Hilbert action but making
sure we don't count any part of the boundary. This is the \textit{bulk
  action}.

We start with the CDT version of the Regge action \cite{Regge} given
by Ambj\o rn \textit{et al.} \cite{DynamicallyTriangulating}:
\begin{eqnarray}
  \label{eq:regge:bare}
  S^{R}_{EH}[\mathcal{T}_{bulk}] &=& k \sum_{\substack{\text{space-like}\\\text{links} \ l}} Vol(l)\frac{1}{i}\left(2\pi - \sum_{\substack{\text{tetrahedra}\\\text{at }l\\t}}\theta_D(t,l)\right) + k \sum_{\substack{\text{time-like}\\\text{links }l}} Vol(l)\left(2\pi - \sum_{\substack{\text{tetrahedra}\\\text{at }l\\t}}\theta_D(t,l)\right)\\
  && \quad - \lambda \sum_{\substack{(3,1)\text{ and }(1,3)\\\text{tetrahedra}}}Vol(3,1)-\lambda\sum_{\substack{(2,2)\\\text{tetrahedra}}}Vol(2,2).\nonumber
\end{eqnarray}
Here $Vol(l)$ is the volume of a given link $l$. Because we are in
$(2+1)$-dimensions. We can assume that $Vol(l)=1$ for space-like links
and that $Vol(l)=\sqrt{\alpha}$ for time-like links. $\theta_D(t,l)$
is the dihedral angle of a tetrahedron $t$ around a link
$l$. $Vol(2,2)$ and $Vol(3,1)$ are the volumes or $(2,2)-$ and
$(3,1)-$ tetrahedra respectively. These values are given in Ambj\o rn
\textit{et al.} \cite{DynamicallyTriangulating}.

If we distribute summation signs and perform obvious summations, we find that:
\begin{eqnarray}
  \label{eq:regge:2}
  S^{R}_{EH}[\mathcal{T}_{bulk}] &=& \frac{2\pi k}{i}\left[N_1^{SL}(\mathcal{T}) - N_1^{SL}(S^{(2)}_i) - N_1^{SL}(S'^{(2)}_f)\right] - \frac{k}{i}\sum_{\substack{\text{space-like}\\\text{links }l}} \ \sum_{\substack{\text{tetrahedra }t\\\text{at link }l}} \theta_D(t,l)\nonumber\\
  &&+2\pi k\sqrt{\alpha} N_1^{TL} - k \sqrt{\alpha} \sum_{\substack{\text{time-like}\\\text{links }l}} \ \sum_{\substack{\text{tetrahedra }t\\\text{at link }l}} \theta_D(t,l)\\
  &&-\lambda \left[V_3^{(3,1)}(N_3^{(3,1)}+N_3^{(1,3)}) + V_3^{(2,2)} N_3^{(2,2)}\right],\nonumber
\end{eqnarray}
where $N_1^{SL}(S^{(2)}_i)$ is the number of space-like links that lie
in the initial surface at proper time $\tau = 0$. $S^{(2)}$ simply
indicates that the topology of the surface is spherical. Likewise
$N_1^{SL}(S'^{(2)}_f$ is the number of space-like links that lie in
the final surface at proper time $\tau=\tau_{final}$. We perform the
subtraction in the first term to avoid overcounting objects in the
boundary multiple times. $V_3^{(3,1)}$ and $V_3^{(2,2)}$ are the
3-volumes of $(3,1)$ (and likewise $(1,3)$) tetrahedra and $(2,2)$
tetrahedra respectively.

We now need to count the number of tetrahedra connected to each link
and sum over all links. To perform this operation, we first act out
the inner sum over dihedral angles around an individual link:
\begin{eqnarray}
  \label{eq:regge:3}
  S^{R}_{EH}[\mathcal{T}_{bulk}] &=& \frac{2\pi k}{i}\left[N_1^{SL}(\mathcal{T}) - N_1^{SL}(\mathcal{S}^{(2)}_i) - N_1^{SL}(\mathcal{S}'^{(2)}_f)\right] - \frac{k}{i}\sum_{\substack{\text{space-like}\\\text{links }l\\\text{in bulk}}} \left[N_3^{(2,2)}(l)\theta_{SL}^{(2,2)} + N_3^{(1,3)}(l)\theta_{SL}^{(3,1)} + N_3^{(3,1)}(l)\theta_{SL}^{(1,3)}\right]\nonumber\\
  &&+2\pi k\sqrt{\alpha} N_1^{TL} - k \sqrt{\alpha} \sum_{\substack{\text{time-like}\\\text{links }l}} \left[N_3^{(2,2)}(l)\theta_{TL}^{(2,2)} + N_3^{(1,3)}(l)\theta_{TL}^{(3,1)} + N_3^{(3,1)}(l)\theta_{TL}^{(3,1)}\right]\\
  &&-\lambda \left[V_3^{(3,1)}(N_3^{(3,1)}+N_3^{(1,3)}) + V_3^{(2,2)} N_3^{(2,2)}\right],\nonumber
\end{eqnarray}
where $N_3^{(2,2)}(l)$, $N_3^{(3,1)}(l)$, and $N_3^{(1,3)}(l)$ are the
number of $(2,2)-$, $(3,1)-$, and $(1,3)-$tetrahedra respectively
around a given link. To perform the remaining summation over the
entire manifold, we look at how many links that each tetrahedron
connects to, and sum over tetrahedrons, rather than summing over
tetrahedrons at each link and then summing over tetrahedra. We know
that:

\begin{itemize}
\item Each $(2,2)-$simplex connects to 2 space-like links in the bulk,
  but only one on each boundary. Thus:
  $$\sum_{\substack{\text{space-like}\\\text{links }l\\\text{in bulk}}} N_3^{(2,2)}(l) = 2 N_3^{(2,2)}(\mathcal{T}_{bulk})$$
  and
  $$\sum_{\substack{\text{space-like}\\\text{links }l\\\text{in boundary}}} N_3^{(2,2)}(l) = N_3^{(2,2)}(\mathcal{S}^{(2)}_i) + N_3^{(2,2)}(\mathcal{S}'^{(2)}_f),$$
  where $N_3^{(2,2)}(\mathcal{T}_{bulk})$ is the total number of
  $(2,2)-$simplices in the manifold that do not connect to a link in
  the boundary. Likewise, $N_3^{(2,2)}(\mathcal{S}^{(2)}_i)$ and
  $N_3^{(2,2)}(\mathcal{S}'^{(2)}_f)$ are the total number of simplices that
  have at least one link (in fact exactly one) in the initial boundary
  or the final boundary respectively. We will continue to use this
  naming convention. Thus:
  \begin{equation}
    \label{eq:22:SL}
      \sum_{\substack{\text{space-like}\\\text{links }l\\\text{in bulk}}} N_3^{(2,2)}(l) = 2 N_3^{(2,2)}(\mathcal{T}_{bulk}) = N_3^{(2,2)} - N_3^{(2,2)}(\mathcal{S}^{(2)}_i) - N_3^{(2,2)}(\mathcal{S}'^{(2)}_f),
  \end{equation}
  where $N_3^{(2,2)}$ is of course the total number of
  $(2,2)$-simplices in the manifold.
\item Each $(2,2)-$simplex connects to 4 time-like links. Thus:
  \begin{equation}
    \label{eq:22:tl}
    \sum_{\substack{\text{time-like}\\\text{links }l}}N_3^{(2,2)}(l) = 4N_3^{(2,2)}.
  \end{equation}
  There are no time-like links in the boundary, so we don't have to
  worry about this distinction.
\item Each $(3,1)-$ and each $(1,3)-$simplex in bulk connects to 3
  spacelike links. Thus
  $$\sum_{\substack{\text{space-like}\\\text{links }l\\\text{in bulk}}}\left(N_3^{(3,1)}+N_3^{(1,3)}\right) 
  = 3\left(N_3^{(3,1)}(\mathcal{T}_{bulk})+N_3^{(1,3)}(\mathcal{T}_{bulk})\right).$$
  On the initial boundary, each $(3,1)-$simplex connects to 3 links,
  but no $(1,3)-$simplex connects to any links at all. Similarly, on
  the final boundary, each $(1,3)-$simplex connects to 3 links but no
  $(3,1)-$simplex connacts to any. Thus:
  $$\sum_{\substack{\text{space-like}\\\text{links }l\\\text{on-boundary}}} (N_3^{(3,1)}(l) + N_3^{(1,3)}(l)) = 3\left(N_3^{(3,1)}(\mathcal{S}^{(2)}_i) + N_3^{(1,3)}(\mathcal{S}'^{(2)}_f\right).$$
  Thus:
  \begin{eqnarray}
    \label{eq:31:sl}
    \sum_{\substack{\text{space-like}\\\text{links }l\\\text{in bulk}}}\left(N_3^{(3,1)}+N_3^{(1,3)}\right) &=& 3\left(N_3^{(3,1)}(\mathcal{T}_{bulk})+N_3^{(1,3)}(\mathcal{T}_{bulk})\right) \nonumber\\
    &=& 3\left(N_3^{(3,1)}+N_3^{(1,3)}\right) - 3\left(N_3^{(3,1)}(\mathcal{S}^{(2)}_i) + N_3^{(1,3)}(\mathcal{S}'^{(2)}_f)\right).
  \end{eqnarray}
\item Each $(3,1)-$ or $(1,3)-$simplex connects to 3 time-like
  links. Thus:
  \begin{equation}
    \label{eq:31:tl}
    \sum_{\substack{\text{time-like}\\\text{links }l}}\left(N_3^{(3,1)}(l)+N_3^{(1,3)}(l)\right) = 3 \left(N_3^{(3,1)}+N_3^{(1,3)}\right).
  \end{equation}
  There are no time-like links in the boundary, so we don't have to
  worry about this distinction.
\end{itemize}

If we take the counting relations given above into account, then we
find that
\begin{eqnarray}
  \label{eq:regge:4}
  S^{R}_{EH}[\mathcal{T}_{bulk}] &=& \frac{2\pi k}{i}\left[N_1^{SL}(\mathcal{T}) - N_1^{SL}(\mathcal{S}^{(2)}_i) - N_1^{SL}(\mathcal{S}'^{(2)}_f)\right] 
   - \frac{k}{i} \theta_{SL}^{(2,2)}\left[ \left(2 N_3^{(2,2)} - N_3^{(2,2)}(\mathcal{S}^{(2)}_i) - N_3^{(2,2)}(\mathcal{S}'^{(2)}_f)\right)\right] \nonumber\\
  &&\qquad - \frac{3k}{i} \theta_{SL}^{(1,3)}\left[ N_3^{(1,3)} + N_3^{(3,1)}) - N_3^{(3,1)}(\mathcal{S}^{(2)}_i) - N_3^{(1,3)}(\mathcal{S}'^{(2)}_f)\right]\\
  && + 2 \pi k\sqrt{\alpha} N_1^{TL} - k \sqrt{\alpha} \left[4 \theta_{TL}^{(2,2)} N_3^{(2,2)} + 3 \theta_{TL}^{(3,1)}\left(N_3^{(3,1)}+N_3^{(1,3)}\right)\right]\nonumber\\
  &&-\lambda \left[V_3^{(3,1)}(N_3^{(3,1)}+N_3^{(1,3)}) + V_3^{(2,2)} N_3^{(2,2)}\right].\nonumber
\end{eqnarray}
This is the bulk form of the Regge action.

\section{The Gibbons-Hawking-York Term}

We now supplement the Regge-Einstein-Hilbert action for
$(2+1)$-dimensional causal dynamical triangulations by the appropriate
Regge-Gibbons-Hawking boundary term. Given the desired spacetime
topology, the boundary $\partial\mathcal{T}_{c}$ consists of two
disconnected components: an initial or past spatial two-sphere
$\mathcal{S}_{i}^{2}$ and a final or future spatial two-sphere
$\mathcal{S}_{f}^{2}$. Based on the demonstration of Hartle and
Sorkin \cite{BoundaryTerm}, we propose the prescription
\begin{equation}
S_{GH}^{(R)}[\partial\mathcal{T}_{c}]=\frac{1}{8\pi G}\sum_{h\in\mathcal{S}_{i}^{2}}\frac{1}{i}\left[\pi-2\theta_{SL}^{(3,1)}-\theta_{SL}^{(2,2)}N_{3\uparrow}^{(2,2)}(h)\right]+\frac{1}{8\pi G}\sum_{h\in\mathcal{S}_{f}^{2}}\frac{1}{i}\left[\pi-2\theta_{SL}^{(1,3)}-\theta_{SL}^{(2,2)}N_{3\downarrow}^{(2,2)}(h)\right].
\end{equation}
Here, $N_{3\uparrow}^{(2,2)}(h)$ is the number of future-directed
$(2,2)$ $3$-simplices attached to the hinge $h$, and
$N_{3\downarrow}^{(2,2)}(h)$ is the number of past-directed $(2,2)$
$3$-simplices attached to the hinge $h$. We justify this prescription
as follows. In parallel transporting the vector normal to one
component of the boundary $\partial\mathcal{T}_{c}$ between two
spacelike $2$-simplices intersecting at the hinge $h$, the vector
rotates through the angle
\begin{equation}
\frac{1}{i}\left[2\theta_{SL}^{(3,1)}+\theta_{SL}^{(2,2)}N_{3}^{(2,2)}(h)\right].
\end{equation}
When this angle is $\frac{\pi}{i}$, the extrinsic curvature vanishes
locally at the hinge $h$; this fact dictates the deficit angle-like
form of our above prescription. The absence of a relative negative
sign between the contributions of the two disconnected components of
the boundary $\partial\mathcal{T}_{c}$ to the Regge-Gibbons-Hawking
boundary term stems from the fact that the future-directed orientation
of the vector normal to $\mathcal{S}_{i}^{2}$ and the past-directed
orientation of the vector normal to $\mathcal{S}_{f}^{2}$ are
accounted for in the past-directed and future-directed orientations of
the $(2,2)$ $3$-simplices attached to the boundary. Performing the
summations over the hinges on the boundary $\partial\mathcal{T}_{c}$,
we may rewrite the Regge-Gibbons-Hawking boundary term as
\begin{eqnarray}
S_{GH}^{(R)}[\partial\mathcal{T}_{c}]&=&\frac{1}{8\pi G}\left[\frac{\pi}{i} N_{1}^{SL}(\mathcal{S}_{i}^{2})-\frac{2}{i}\theta_{SL}^{(3,1)}N_{1}^{SL}(\mathcal{S}_{i}^{2})-\frac{1}{i}\theta_{SL}^{(2,2)}N_{3\uparrow}^{(2,2)}(\mathcal{S}_{i}^{2})\right]\nonumber\\ &&+\frac{1}{8\pi G}\left[\frac{\pi}{i} N_{1}^{SL}(\mathcal{S}_{f}^{2})-\frac{2}{i}\theta_{SL}^{(3,1)}N_{1}^{SL}(\mathcal{S}_{f}^{2})-\frac{1}{i}\theta_{SL}^{(2,2)}N_{3\downarrow}^{(2,2)}(\mathcal{S}_{f}^{2})\right].
\end{eqnarray}

The complete Regge action is thus
\begin{eqnarray}
S^{(R)}[\mathcal{T}_{c}]&=& \frac{2\pi k}{i}\left[N_1^{SL}(\mathcal{T}) - N_1^{SL}(\mathcal{S}^{(2)}_i) - N_1^{SL}(\mathcal{S}'^{(2)}_f)\right] 
   - \frac{k}{i} \theta_{SL}^{(2,2)}\left[ \left(2 N_3^{(2,2)} - N_3^{(2,2)}(\mathcal{S}^{(2)}_i) - N_3^{(2,2)}(\mathcal{S}'^{(2)}_f)\right)\right] \nonumber\\
  &&\qquad - \frac{3k}{i} \theta_{SL}^{(1,3)}\left[ N_3^{(1,3)} + N_3^{(3,1)}) - N_3^{(3,1)}(\mathcal{S}^{(2)}_i) - N_3^{(1,3)}(\mathcal{S}'^{(2)}_f)\right]\nonumber\\
  && + 2 \pi k\sqrt{\alpha} N_1^{TL} - k \sqrt{\alpha} \left[4 \theta_{TL}^{(2,2)} N_3^{(2,2)} + 3 \theta_{TL}^{(3,1)}\left(N_3^{(3,1)}+N_3^{(1,3)}\right)\right]\nonumber\\
  &&-\lambda \left[V_3^{(3,1)}(N_3^{(3,1)}+N_3^{(1,3)}) + V_3^{(2,2)} N_3^{(2,2)}\right].\\
  &&+\frac{1}{8\pi G}\left[\frac{\pi}{i} N_{1}^{SL}(\mathcal{S}_{i}^{2})-\frac{2}{i}\theta_{SL}^{(3,1)}N_{1}^{SL}(\mathcal{S}_{i}^{2})-\frac{1}{i}\theta_{SL}^{(2,2)}N_{3\uparrow}^{(2,2)}(\mathcal{S}_{i}^{2})\right]\nonumber\\ 
  &&+\frac{1}{8\pi G}\left[\frac{\pi}{i} N_{1}^{SL}(\mathcal{S}_{f}^{2})-\frac{2}{i}\theta_{SL}^{(3,1)}N_{1}^{SL}(\mathcal{S}_{f}^{2})-\frac{1}{i}\theta_{SL}^{(2,2)}N_{3\downarrow}^{(2,2)}(\mathcal{S}_{f}^{2})\right].\nonumber
\end{eqnarray}

\section{Consistency Checks}

We finally demonstrate that our prescription for the
Regge-Gibbons-Hawking boundary term in $(2+1)$-dimensional causal
dynamical triangulations is consistent with the form of the
Regge-Einstein-Hilbert action determined by Ambj\o rn \emph{et al}. We
make such a demonstration by verifying that our prescription for the
Regge-Gibbons-Hawking boundary term reproduces the
Regge-Einstein-Hilbert action when we compose two spacetime regions
sharing a common boundary $\mathcal{S}_{c}^{2}$. Consider two
triangulated spacetime manifolds $\mathcal{T}_{c}$ and
$\mathcal{T}_{c}'$ both with two-sphere spatial topology and line
interval temporal topology. The boundary $\partial\mathcal{T}_{c}$
consists of an initial two-sphere $\mathcal{S}_{i}^{2}$ and a final
two-sphere $\mathcal{S}_{f}^{2}$, and the boundary
$\partial\mathcal{T}_{c}'$ consists of an initial two-sphere
$\mathcal{S'}_{i}^{2}$ and a final two-sphere
$\mathcal{S'}_{f}^{2}$. To compose the two triangulated spacetime
manifolds $\mathcal{T}_{c}$ and $\mathcal{T}_{c}'$, we first take the
two-spheres $\mathcal{S}_{f}^{2}$ and $\mathcal{S'}_{i}^{2}$ to have
the same intrinsic geometry and then we orient the two-spheres
$\mathcal{S}_{f}^{2}$ and $\mathcal{S'}_{i}^{2}$ to have coincident
normal vectors. We may thus identify these two two-spheres as
$\mathcal{S}_{c}^{2}$. The Regge-Gibbons-Hawking boundary term
contributions of $\mathcal{S}_{c}^{2}$ from the two triangulated
spacetime manifolds $\mathcal{T}_{c}$ and $\mathcal{T}_{c}'$ are
\begin{eqnarray}
\frac{1}{8\pi G}\left[\frac{\pi}{i} N_{1}^{SL}(\mathcal{S}_{f}^{2})-\frac{2}{i}\theta_{SL}^{(3,1)}N_{1}^{SL}(\mathcal{S}_{f}^{2})-\frac{1}{i}\theta_{SL}^{(2,2)}N_{3\downarrow}^{(2,2)}(\mathcal{S}_{f}^{2})\right]\nonumber\\+\frac{1}{8\pi G}\left[\frac{\pi}{i} N_{1}^{SL}(\mathcal{S'}_{i}^{2})-\frac{2}{i}\theta_{SL}^{(3,1)}N_{1}^{SL}(\mathcal{S'}_{i}^{2})-\frac{1}{i}\theta_{SL}^{(2,2)}N_{3\uparrow}^{(2,2)}(\mathcal{S'}_{i}^{2})\right].
\end{eqnarray}
Together these two Regge-Gibbons-Hawking boundary terms combine to give the contribution to the Regge-Einstein-Hilbert action coming from the spacelike hinges on $\mathcal{S}_{c}^{2}$.

%Bibliography
\bibliography{CDT_boundaries}
\bibliographystyle{unsrt}

\end{document}